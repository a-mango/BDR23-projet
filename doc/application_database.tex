\documentclass{article}

% \usepackage[latin1]{inputenc}
\usepackage[french]{babel}
\usepackage[T1]{fontenc}

\usepackage{lastpage}
\usepackage{fancyhdr}
\usepackage{graphicx}
\usepackage{pdflscape}

\usepackage[a4paper, margin=2.2cm, footskip=12.3pt]{geometry}

\setcounter{secnumdepth}{0}

\newcommand{\header} {
    \setlength{\headheight}{30pt}\pagestyle{fancy}
    \fancyhead[L]{\includegraphics[height=20pt]{../assets/logo.pdf}}\fancyhead[C]{}
    \fancyhead[R]{Vitória Cosmo, Aubry Mangold, Eva Ray\\\today}\fancyfoot[C]{}
    \fancyfoot[R]{Page \thepage~sur \pageref{LastPage}}\renewcommand{\footrulewidth}{0.3pt}
}

\title{Gestion d'un service de réparation d'objets\\[1ex]Application de base de données}
\author{Vitória Cosmo, Aubry Mangold, Eva Ray}
\date{21 janvier 2024}

\begin{document}
\header

\maketitle

\section{Architecture}
L'application contient trois composants principaux: une base de données, une API et un site web. Le frontend communique avec l'API qui, à son tour, interroge la base de données afin de récupérer
les informations. L'API traite la réponse de la base de données au frontend, qui l'affiche.  

\section{Technologies}

\begin{itemize}
\item La base de données est construite et peut être interrogée en utilisant la language \texttt{postgresql}.
\item L'API a été codée en \texttt{Java} à l'aide du web framework \texttt{Javalin}. Le connecteur \texttt{JDBC} postgresql est utilisé pour se connecter à l'instance de la base de donnée.
\item Le frontend a été développé avec \texttt{React}. Les requêtes sont effectuées avec la librairie \texttt{Axios} 
\item \texttt{Docker compose} est utilisé pour définir l'environnement nécessaire au fonctionnement de l'application.
\end{itemize}
    
\section{Requêtes}
Les requêtes de l'API sur la base de données sont effectuées en utilisant des \textit{prepared statement}. Le résultat des requêtes est transformé en \texttt{POJO} (\textit{Plain Old Java Object}) qui sont ensuite sérialisés puis envoyés par Javalin au travers du \texttt{Context}.

\section{Fonctionnalités}
L'application dispose des fonctionnalités suivantes:
\begin{itemize}
\item Des statistiques concernant la base données qui sont consultables dans le dashboard se trouvant sur la page \texttt{home} du site web.
\item Un formulaire disponible sur le site web permettant d'ajouter et de mettre à jour des réparations.
\item Un formulaire disponible sur le site web permettant d'ajouter et de mettre à jour des customers.
\item Des données concernant les réceptionistes, techniciens, managers, clients, et réparations sont affichées sur leurs pages respectives sur le site.
\item Sur le site web, trois sections différentes selon le rôle que prend l'utilisateur: technicien, réceptionniste et manager. Un technicien a accès aux réparations. 
Un réceptionniste a accès aux clients et aux réparations. Un manager a accès à toutes les pages disponibles. Chaque utilisateur a accès au dashboard, indépendemment de son rôle.
\end{itemize}

\section{Axes de développement}
\begin{itemize}
\item Ajouter des formulaires sur le site web permettant de gérer les réceptionistes et les managers.
\item Ajouter une section permettant de consulter et gérer les objets en vente sur le site web.
\item Ajouter une page permettant de visualiser et de gérer le temps passé par les techniciens sur les réparations.
\end{itemize}

\end{document}