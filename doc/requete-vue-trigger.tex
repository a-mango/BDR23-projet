\documentclass{article}

% \usepackage[latin1]{inputenc}
\usepackage[french]{babel}
\usepackage[T1]{fontenc}
\usepackage{lastpage}
\usepackage{fancyhdr}
\usepackage{graphicx}
\usepackage{pdflscape}
\usepackage[a4paper, margin=2.2cm, footskip=12.3pt]{geometry}

\setcounter{secnumdepth}{0}

\newcommand{\header} {
    \setlength{\headheight}{30pt}\pagestyle{fancy}
    \fancyhead[L]{\includegraphics[height=20pt]{../assets/logo}}\fancyhead[C]{}
    \fancyhead[R]{Vitória Cosmo, Aubry Mangold, Eva Ray\\\today}\fancyfoot[C]{}
    \fancyfoot[R]{Page \thepage~sur \pageref{LastPage}}\renewcommand{\footrulewidth}{0.3pt}
}

\newcommand{\ul}{\underline}
\newcommand{\ttt}{\texttt}

\title{Gestion d'un service de réparation d'objets\\[1ex]Requêtes, vues, triggers et données}
\author{Vitória Cosmo, Aubry Mangold, Eva Ray}
\date{14 décembre 2023}

\begin{document}
    \header


    \maketitle


    \section{Requêtes}

    Les requêtes suivantes ont été créées afin de manipuler la base de données :
    \begin{itemize}
        \item Consulter les réparations pour un technicien donné.
        \item Modifier l'état d'une réparation.
        \item Modifier la description d'une réparation.
        \item Ajouter un temps de travail à une réparation.
        \item Consulter les données d'un client.
        \item Modifier les données d'un client.
        \item Consulter les données d'une réparation.
        \item Créer une réparation.
        \item Modifier les données d'une réparation.
        \item Modifier l'état du devis d'une réparation.
        \item Consulter les données d'une vente.
        \item Créer une vente.
        \item Modifier les données d'une vente.
        \item Envoyer un SMS.
        \item Modifier l'état d'un SMS.
        \item Consulter un SMS.
        \item Consulter les données d'un échange SMS lié à une réparation donnée dans l'ordre antéchronologique.
        \item Modifier les données d'un collaborateur.
        \item Effacer une personne.
        \item Effacer une vente.
        \item Dans l'API, des requêtes basiques pour mettre en place les opérations CRUD (SELECT, INSERT, UPDATE, DELETE) sur les différentes tables ont été créées.
    \end{itemize}

    Les requêtes suivantes ont été réalisées à de fins statistiques :
    \begin{itemize}
        \item Obtenir la quantité d'employés par rôle.
        \item Obtenir la quantité de réparations en cours.
        \item Obtenir la quantité de réparations par catégorie.
        \item Obtenir la quantité de réparations terminées.
        \item Obtenir la quantité de réparations terminées par catégorie.
        \item Obtenir la quantité d'objets en vente.
        \item Obtenir la quantité d'objets vendus.
        \item Obtenir la quantité d'objets traités par catégorie.
        \item Obtenir la quantité d'objets traités par marque.
        \item Obtenir la quantité d'heures travaillées par spécialisation.
        \item Obtenir la quantité de réparations traitées par mois.
        \item Obtenir la quantité de réparations créées par réceptionniste.
        \item Obtenir la quantité de réceptionnistes par langue.
        \item Obtenir la quantité de SMS reçus par jour.
        \item Obtenir la quantité de SMS envoyés par jour.
    \end{itemize}

    \pagebreak

    \section{Vues}

    Les vues suivantes ont été créées :
    \begin{itemize}
        \item \ttt{collab\_role\_id\_view} : affiche les id des collaborateurs et leur rôle.
        \item \ttt{works\_on} : liste les réparations en cours et le temps travaillé pour un technicien donné.
        \item \ttt{technician\_view} : affiches les informations des réparations auquel un technicien est lié.
        \item \ttt{collab\_info} : affiche les infos d'un collaborateur.
        \item \ttt{receptionist\_view} : affiche les infos d'un réceptionniste.
        \item \ttt{customer\_info\_view} : affiche les infos d'un client.
    \end{itemize}


    \section{Triggers}

    Les triggers suivants ont été créés :
    \begin{itemize}
        \item \ttt{on\_reparation\_update\_update\_date} : met à jour le champ \ttt{date\_modified} à la date actuelle lors d'une mise à jour d'une réparation.
        \item \ttt{verify\_quote\_state\_is\_declined\_for\_reparation} : vérifie que le champ \ttt{quote\_state} est bien à \ttt{declined} lors d'une insertion dans la table \ttt{sale}.
        \item \ttt{verify\_quote\_state\_is\_declined\_for\_object} : vérifie que le champ \ttt{quote\_state} est bien à \ttt{declined} lorsque le champ \ttt{location} est mis à jour en \ttt{for\_sale} ou \ttt{sold}.
        \item \ttt{verify\_quote\_state\_is\_accepted} : vérifie que le champ \ttt{quote\_state} est bien à \ttt{accepted} lorsque le champ \ttt{reparation\_state} de réparation est mis à jour en \ttt{ongoing} ou \ttt{done}.
        \item \ttt{verify\_tos\_accepted} : vérifie que les \ttt{tos} sont bien acceptés lors de la création d'une réparation.
        \item \ttt{set\_reservation\_state\_ongoing} : met à jour le champ \ttt{reparation\_state} en \ttt{ongoing} lorsque le champ \ttt{quote\_state} de la réparation est mis à jour en \ttt{accepted}.
        \item \ttt{before\_insert\_receptionist\_language} : si un langage n'est pas présent dans la table \ttt{language}, lors d'une insertion dans la table de jointure \ttt{receptionist\_language}, le langage est inséré dans la table \ttt{language}.
        \item \ttt{update\_collaborator\_person\_trigger} : update les tables \ttt{collaborator} et \ttt{person} lors d'une mise à jour de la view \ttt{collab\_info\_view}.
        \item \ttt{update\_collaborator\_person\_trigger\_on\_insert} : update les tables \ttt{collaborator} et \ttt{person} lors d'une insertion dans la view \ttt{collab\_info\_view}.
        \item \ttt{delete\_collaborator\_person_trigger\_on\_delete} : supprime les lignes des tables \ttt{collaborator} et \ttt{person} lors d'une suppression dans la view \ttt{collab\_info\_view}.
        \item \ttt{update\_customer\_person\_trigger} : update les tables \ttt{customer} et \ttt{person} lors d'une mise à jour de la view \ttt{customer\_info\_view}.
        \item \ttt{update\_customer\_person\_trigger\_on\_insert} : update les tables \ttt{customer} et \ttt{person} lors d'une insertion dans la view \ttt{customer\_info\_view}.
        \item \ttt{delete\_customer\_person\_trigger\_on\_delete} : supprime les lignes des tables \ttt{customer} et \ttt{person} lors d'une suppression dans la view \ttt{customer\_info\_view}.
        \item \ttt{update\_collaborator\_person\_receptionist\_trigger} : update les tables \ttt{collaborator}, \ttt{person} et \ttt{receptionist} lors d'une mise à jour de la view \ttt{receptionist\_view}.
        \item \ttt{update\_collaborator\_person\_receptionist\_trigger\_on\_insert} : update les tables \ttt{collaborator}, \ttt{person} et \ttt{receptionist} lors d'une insertion dans la view \ttt{receptionist\_view}.
        \item \ttt{delete\_collaborator\_person\_receptionist\_trigger\_on\_delete} : supprime les lignes des tables \ttt{collaborator}, \ttt{person} et \ttt{receptionist} lors d'une suppression dans la view \ttt{receptionist\_view}.
        \item \ttt{update\_collaborator\_person\_technician\_trigger} : update les tables \ttt{collaborator}, \ttt{person} et \ttt{technician} lors d'une mise à jour de la view \ttt{technician\_view}.
        \item \ttt{update\_collaborator\_person\_technician\_trigger\_on\_insert} : update les tables \ttt{collaborator}, \ttt{person} et \ttt{technician} lors d'une insertion dans la view \ttt{technician\_view}.
        \item \ttt{delete\_collaborator\_person\_technician\_trigger\_on\_delete} : supprime les lignes des tables \ttt{collaborator}, \ttt{person} et \ttt{technician} lors d'une suppression dans la view \ttt{technician\_view}.
        \item \ttt{update\_collaborator\_person\_manager\_trigger} : update les tables \ttt{collaborator}, \ttt{person} et \ttt{manager} lors d'une mise à jour de la view \ttt{manager\_view}.
        \item \ttt{update\_collaborator\_person\_manager\_trigger\_on\_insert} : update les tables \ttt{collaborator}, \ttt{person} et \ttt{manager} lors d'une insertion dans la view \ttt{manager\_view}.
        \item \ttt{delete\_collaborator\_person\_manager\_trigger\_on\_delete} : supprime les lignes des tables \ttt{collaborator}, \ttt{person} et \ttt{manager} lors d'une suppression dans la view \ttt{manager\_view}.
    \end{itemize}

    \pagebreak

    \section{Procédures}

    Les procédures suivantes ont été créées :
    \begin{itemize}
        \item \ttt{InsertCustomer} : insère un client dans la base de données.
        \item \ttt{InsertCollaborator} : insère un collaborateur dans la base de données.
        \item \ttt{InsertManager} : insère un manager dans la base de données.
        \item \ttt{InsertReceptionist} : insère un réceptionniste dans la base de données.
        \item \ttt{InsertTechnician} : insère un technicien dans la base de données.
        \item \ttt{createReceptionist} : crée un nouveau réceptionniste (utilisée par l'api)
        \item \ttt{updateReceptionist} : met à jour un réceptionniste (utilisée par l'api)
        \item \ttt{createReparation} : crée une nouvelle réparation (utilisée par l'api)
    \end{itemize}


    \section{Données}

    Des données ont été insérées dans toutes les tables de la base de données. Les points suivants sont relevants :
    \begin{itemize}
        \item 50 clients, 3 managers,15 techniciens et 3 réceptionnistes sont insérés.
        \item 50 réparations et objets ont été insérées. Chaque réparation est liée à un client différent. Les différents états des énumérations \ttt{quote\_state}, \ttt{reparation\_state} et \ttt{location} sont utilisés.
        \item 100 SMS ont été insérés. Chaque réparation est liée à un couple de SMS. Les SMS sont tous dans l'état \ttt{processed}.
        \item 6 ventes ont été insérées. Certaines des ventes sont terminées.
        \item Les spécialités, catégories, marques, languages et autres tables sont remplis.
    \end{itemize}

\end{document}
